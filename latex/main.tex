\documentclass[a4paper,11pt]{jsarticle}


% 数式
\usepackage{amsmath,amsfonts}
\usepackage{bm}
% 画像
\usepackage[dvipdfmx]{graphicx}


\begin{document}

\title{応用微分方程式}
\author{Nanase Takahashi}
\date{\today}
\maketitle

\section{問題}

\begin{equation}\label{problem}
    \frac{d^2}{dz^2}w(z) - zw(z) = 0
\end{equation}

の$z=0$周りの任意の複素数$z$に対する解を$w(z)=\sum_{n=0}^{+\infty}c_nz^n$と仮定し,与えられた微分方程式に代入することにより一般解を求めよ.

\section{解答}
仮定より,
\[
w(z)=\sum_{n=0}^\infty c_n\,z^n,
\quad
w''(z)=\sum_{n=2}^\infty n(n-1)\,c_n\,z^{n-2}
=\sum_{m=0}^\infty (m+2)(m+1)\,c_{m+2}\,z^m.
\]
これを(\ref{problem})式に代入すると,
\[
\sum_{m=0}^\infty (m+2)(m+1)\,c_{m+2}\,z^m
\;-\;
\sum_{n=0}^\infty c_n\,z^{n+1}
\;=\;0.
\]
後者の総和を添字 $m=n+1$ で書き直すと
\[
\sum_{m=1}^\infty c_{m-1}\,z^m,
\]
したがって
\[
(m+2)(m+1)\,c_{m+2}-c_{m-1}=0
\quad(m=1,2,\dots),
\quad
2\cdot1\,c_2=0.
\]
よって
\[
c_2=0,
\quad
c_{m+2}=\frac{c_{m-1}}{(m+2)(m+1)}
\quad(m\ge1).
\]

初期係数として $c_0$ および $c_1$ は任意定数となる.
漸化式を見ると,係数は3つ飛びで定まるため,
\[
c_3=\frac{c_0}{3\cdot2},\quad
c_6=\frac{c_3}{6\cdot5}=\frac{c_0}{(2\cdot3)(5\cdot6)},\;\dots
\]
一方,
\[
c_4=\frac{c_1}{4\cdot3},\quad
c_7=\frac{c_4}{7\cdot6}=\frac{c_1}{(3\cdot4)(6\cdot7)},\;\dots
\]
となる.したがって,
\begin{equation}
  w(z)=A(1+\tfrac{1}{2\cdot3}z^3+\tfrac{1}{2\cdot3\cdot5\cdot6}z^6+\cdots)
  + B(z+\tfrac{1}{3\cdot4}z^4+\tfrac{1}{3\cdot4\cdot6\cdot7}z^7+\cdots)
\end{equation}

ここで $A,B$ は任意定数である.
\end{document}
